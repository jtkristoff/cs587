\documentclass[sigconf]{acmart}
\bibliographystyle{ACM-Reference-Format}

\begin{document}

\title{Tor Fingerprinting Attacks}
\author{John Kristoff}
\email{jtk@depaul.edu}

\begin{abstract}

Tor, the anonymity service, is in a constant game of leap frog between
Tor users and would be censors.  Tor helps not only protect
communications through encryption from would be eavesdroppers, but helps
limit the identification of the ends of the communications, providing
more than just secrecy, but also anonymity.  Censors, unless they wish
to completely dismantle all communications must be able to detect and
limit access to certain information.  Some censors not only wish to
limit the flow of prohibited communication, but may seek to identify and
possibly punish those who would subvert their control.  Tor, it is no
secret, is no friend to would-be censors and to those who would prefer
to identify the users trying to circumvent their controls.  Censors seek
to attack the Tor system by identifying the components and its users.
Affecting the availability of the Tor system begins by fingerprinting
its components parts, the relay nodes, network traffic, and end users.
This paper examines the available literature on Tor system
fingerprinting and contributes some measurement-based perspective as the
system is commonly deployed and used.

\end{abstract}

\maketitle

\section{Introduction}

Tor is an overlay network for the Internet that aims to provide
anonymity for users and services.\cite{dingledine_tor:_2004}  Using a
technology known as \emph{onion routing} that grew out of research
sponsored in part by the United States Office of Naval Research and
DARPA, Tor has become one of the best known and most widely used
anonymity services on the the Internet today.  The public Tor network
consists of approximately 6,600 relay nodes, 100 Gb/s of relayed
traffic, and 3 million connected clients.\footnote{Tor relay node
statistics from data provided by \url{https:/torstatus.blutmagie.de/}
and usage statistics as displayed at
\url{https://metrics.torproject.org/}, both as listed in December 2017.}

Access to the Tor network is relatively straightforward.  Using free and
widely available software, Tor software is available for an array of
operating systems and environments.  In order to participate in the Tor
network a well known list of systems must be published so that relays
and users alike can learn how to bootstrap into the system and build
circuits through a reasonable set of relays out of a larger available
pool.  Since this priming list of relay nodes must be widely available
and public, identifying relay nodes is straightforward.  Likewise,
identifying users of the Tor system can be a matter of simply observing
traffic between a known target endpoint and the known list of Tor
relays.  Of course there is still a matter of the traffic being
encrypted that may pose an additional challenge for the would be
attacker.

Obtaining a fresh copy of the public list of Tor relay nodes would seem
to be sufficient to launching a fingerprint-based attack.  More
sophisticated approaches of identifying components of the Tor system
might be unnecessary.  However, there are at least four situations where
this bootstrapping list is insufficient.  First, an attacker may not
have the means to inspect traffic of a target system, either at the
relay node or where an end user is positioned.  Second, when a Tor relay
node is co-resident with other, possibly unrelated but widely used
services, falsely associating a conversation as Tor communications may
occur.  Third, a Tor relay node may not be widely known, such as the
case with Tor \emph{bridge} nodes or when used by private, unlisted Tor
networks.  Fourth, there may be the desire to go beyond simple Tor
system usage fingerprinting and uncover end-to-end application and
content, effectively breaking the anonymity protections Tor was designed
to provide.  This last attack type is where most of the research
activity has been focused, often in the form of traffic analysis attack
models, some of which we summarize in the next section.  While
interesting and useful work, we are also interested in general Tor
traffic and system usage fingerprinting covering the first three
scenarios.  We provide some evaluation and ideas from some ongoing work
with our own measurement and analysis infrastructure in \S
\ref{sec:our_contributions} and \S \ref{sec:future_work}.

\section{Literature Survey}\label{sec:literature_survey}

\subsection{Early Anonymity Attacks}\label{subsec:early_anonymity_attacks}

Much of the anonymity fingerprinting literature is focused on uncovering
the \emph{identities} of a pair of communicating parties.  Network
anonymity systems are typically designed with one or more relays as a
means of ensuring at least one end of the conversation is not directly
aware of the identify on the other end in a conversation, providing
anonymity to the initiating end of the conservation.  A series of one or
more anonymous relays should be unable and unwilling to disclose an
initiating end's identity to the destination.  Attacks on these trust
relationships date back before the current Tor system was fully
deployed.  We begin by briefly outlining the first attacks against
anonymity, which provide lessons for a bevy of relay-based anonymity
systems such as Tor, foreshadowing much of the attack research to come.

Syverson et al., 2001 \cite{syverson_towards_2001} provide the first
known mathematical analysis of compromised relays in an onion routing
system.  While compromised relays are beyond the scope of this paper,
they provide some helpful commentary on the onion routing network model,
which eventually spawned into the Tor project.  They eluded to the
possibility of "potentially trackable" changes in link traffic patterns.
They assumed that relays would be configured to operate in a symmetric
manner, minimizing differences that may lead to reducing the
predictability of traffic patterns.  It turns out in practice, many Tor
relay operators implement a varied set of policies, from disallowing a
relay from acting as an \emph{exit} node to limiting which applications
may be permitted to transit an exit relay.

Back et al., 2001 \cite{back_traffic_2001} described a series of basic
traffic analysis attacks against the Freedom network.\footnote{The
Freedom network is a now defunct anonymity network of relays owned and
operated by a company then known as Zero Knowledge Systems (ZKS).  ZKS
built and deployed was could arguably be called an early precursor, very
similar in concept and operation, to the modern day Tor network.}  One
of their attacks, anticipating an approach to more advance attacks to
come, was a simple packet counting correlation attack.  This attack
requires being able to observe traffic entering and leaving relays.
Even if relays are shared, this attack attempts to correlate a series of
packets on ingress to a relay with an equivalent set of packets on
egress.  If you're able to observe all of the anonymity network relays,
you could, at least in theory, uncover the two ends of a conversation
through this packet counting correlation.  Hintz, 2002
\cite{hintz_fingerprinting_2002} developed a similar early rudimentary
attack using traffic byte counting.  While packet-based fingerprinting
attacks are still a potential threat, the byte-counting approach has
been defeated by Tor's use of fixed size \emph{cells}.  While the Tor
project has long been aware of packet counting attacks, defenses to them
have not become an integral part of the Tor system to this day.  Packet
counting attacks remain an area of ongoing threat research.

In Dingledine et al., 2004 \cite{dingledine_tor:_2004} the authors
detail the design of the modern Tor system.  From design goals to
reflections on attacks and defenses, the authors lay out a number of
assumptions and decisions that has led to basis for the modern-day Tor
system in widespread use.  This paper highlights trade offs made, such as
those between protecting traffic from analysis and system complexity.
The Tor system maintainers have been particularly transparent and
forthcoming about the potential threats and weaknesses of the overall
system both in this paper and on their blog.  For example, see
\cite{dingledine_research_2011}.  This paper notes several possible
avenues of attack through traffic analysis and relay node manipulation.
The general theme of this paper is while there are many avenues of
attack, the system is designed to be reasonably resistant to casual
attacks through network size, diversity, and adherence to best
practices.  In other words, a determined attacker is likely able to
compromise anonymity given sufficient motivation and resources.

\subsection{Website Fingerprinting}\label{subsec:website_fingerprinting}

In Shi \& Matsuura, 2009 \cite{shi_fingerprinting_2009} a traffic
observation attack between an end user is described.  The authors
classify a stream of related packets into \emph{vectors}.  Each vector
is a unidirectional packet flow within a time-limited interval.  Some
characteristics such as packet count, size, and TCP window size can
inform the classifier.  The vectors must first be preclassified and then
matched with an ingress or egress flow of Tor system packets as they
enter and exit the overlay network.  By classifying web traffic to
popular sites and matching preseeded patterns with test traffic they
were able to identify traffic endpoints with a high degree of
probability when the end user was limited to a small set of known sites.
As the variety of sites increased success fell off dramatically. (i.e.
after only about 100 distinct sites were evaluated the success rate was
limited to approximately 50\%).  This paper presented a relatively
rudimentary experimental approach to traffic classification with only a
marginal rate of success.  As destination diversity and web page
uniqueness increases the attack is likely to fare even worse.

Pachenko et al. 2011 \cite{panchenko_website_2011} introduced novel
machine learning using support vector machine (SVM) techniques to
website fingerprinting (WFP) and summarized the state of the art of
attacks on anonymization networks up to that point in time.  The authors
advanced the traffic analysis approach noticeably, going beyond simple
byte and packet counting, giving rise to a number of successor
approaches using machine learning algorithms.  One innovation introduced
in this paper was the use of \emph{markers}, which act like boundaries
of a Poisson distribution, though the authors make no mention of this
statistical intention.  Markers delineate size of packets, directional
flow, HTML objects and packet counts.  Classification using statistical
distributions of these marker-based profiles helps Pachenko et al.
reach fingerprinting rates of slightly more than 54\% on average.

Inevitably traffic fingerprinting and analysis was destined to rise
above early attempts that focused on relatively simpler packet and byte
counting mechanisms.  In Wang \& Goldberg, 2013
\cite{wang_improved_2013}, such advanced techniques were developed.
Like most traffic fingerprinting attacks, observation of a target's
ingress and egress traffic to the Tor system is required.  In this case
the authors find that by reducing this traffic to exclude Tor control
traffic, specifically \texttt{RELAY\_SENDME} cells.  This seemingly
minor optimization helps the authors advance the success rate of traffic
identification from 86-87\% to 91\% of earlier studies, while keeping
the false positive rate to a fraction of a percent.  While these modest
attack gains seem negligible, the authors significantly increased the
sophistication of the attack methods over most previous methods.

Wang \& Goldberg advanced the start of the art in at least three new
ways.  First, the authors leveraged the ability to steer Tor circuit
construction, choosing the series of relays a path of packets follow.
This allows them to study and hone the ability to classify traffic, as
well as detect any differences in exit choice.  Depending on the exit
relay, traffic may be directed to one of a set of destination instances
when a web site is globally distributed through load balancing.  While
many times the destination may be respond with exact replicas of content
when distributed, often times their behavior may assume distinct
behavior due to presumed \emph{localization} effects.  For instances, a
web site may present text in a web page in a language associated with
the geolocated position of the connecting source address.

Second, the authors consider and evaluate traffic pattern
characteristics on the basis of Tor cells, rather than just the
underlying encapsulated IP traffic.  This important distinction allows
the authors to extract entropy based on Tor-specific overlay network
behavior that is lost when considering TCP/IP packet detail alone.
While Tor may reveal underlying TCP/IP network fingerprinting through
analysis, considering Tor's use of cells and control traffic provides a
potentially rich avenue for additional insight.  As noted above,
removing the \texttt{RELAY\_SENDME} cells enhances the ability to
successfully perform traffic fingerprinting.

Third, the authors further machine learning techniques, using a SVM
approach, to classify and fingerprint web traffic.  While earlier attack
research also used SVMs, Wang \& Goldberg refine the the learning
algorithms to both widen the breadth of similarity between training data
and reduce seemingly unnecessary tight controls of differences in
dynamic web characteristics such as advertisements.

Rounding out a look at website fingerprinting attack literature, He et
al., 2014 \cite{he_novel_2014} go beyond passive inspection to actively
introducing client latency to artificially improve analysis on these
more predictable traffic flows.  In other words, by isolating and
separating what are often multiple website connections to obtain a
variety of site objects, He et al. leverage this behavior to enhance a
SVM-based classification engine.  They are able to achieve a 15\%
detection advantage over prior methods using this approach.  The
drawback of this system is the delicacy required to introduce latency
without either disrupting communications or being detected when
communications are sufficiently changed or degraded.  It is also worth
noting that this approach must be carefully aligned with expected
retransmission timers and traffic profiles in order to remain effective.

\subsection{Onion Services Identification}

Fingerprint attacks on Tor onion services (formerly known as hidden
services) and onion service users has become an active area of research
in recent years thanks in part to the rise and fall of the anonymous,
online underground market known as \emph{Silk Road}.\footnote{The
original Silk Road service was shutdown by the FBI in October 2013, and
the so-called Silk Road 2.0 successor was also shutdown by the FBI in
cooperation with Europol in 2014.}  As Silk Road may have most famously
demonstrated, Tor can be used not only to provide anonymity to users of
common services such as the world wide web, but Tor can also provide
anonymity to web sites and other server applications.  Consequently,
researchers have sought out attacks to identify and fingerprint these
hidden services and their users.

The first known attack analysis on hidden services was done by
{\O}verlier \& Syverson, 2006 \cite{overlier_locating_2006} and predates
the extraordinary events of the \emph{Silk Road} market.  Unlike most
surveyed literature we have examined at this point, this paper is unique
in that not only does it prevent noteworthy attack research, but it led
to changes in the Tor system by the time of publication.  Therefore, we
present just a cursory look as this attack where a derivative form is no
longer practical.  These attacks, as do many of those we look at require
an attacker be in control of one of the Tor relay nodes.  The first of
four attacks takes advantage of publication or non-publication of a
hidden service relay.  Monitoring the hidden service and the relays
listed in the Tor directory can be used to detect which relay is
responsible for a hidden service when both the directory entry and the
hidden service becomes unavailable.  Two other attacks undermine the
ability to infer conversations by examining the timing of communications
flowing through a monitored relay.  The fourth and final attack is aimed
at rendezvous points (RPs), intermediary tor relay nodes each end of Tor
circuit rely upon to meet and exchange messages through.  An attacker
controlling a relay node and a RP will give an attacker knowledge an
advantage when their relay node is the last node between one of the
hidden service endpoints.

Kwon et al., 2015 \cite{kwon_circuit_2015} first realized Tor circuits
established for hidden services are setup and maintained separately than
for general purpose Tor relay traffic.  This leads to an obvious
reduction in traffic analysis required to fingerprint hidden services.
A relay operator, while presumably unable to decrypt any of the Tor
network traffic or identify both endpoints of a typical Tor circuit,
nonetheless has access to the forward and reverse next-hop circuit
nodes.  This allows an attack to classify circuits, particularly
circuits used for hidden services.  Coupled with traditional website
fingerprinting attacks it then becomes possible to fingerprint and
identify end-to-end hidden service endpoints.  For endpoints that have
been trained through prior traffic, fingerprint detection rates can be
over 90\%!  However, this is an optimistic measure for well known or
popular endpoints, with actual results in real traffic scenarios being
far less in practice.

Overdorf et al., 2017 \cite{overdorf_how_2017} took web fingerprinting
attacks on hidden services and adapted them to the "closed world" of Tor
\texttt{.onion} sites.\footnote{The \texttt{.onion} top-level domain is
a DNS-like name space specific to the Tor system.  Hidden services use
the \texttt{.onion} domain name preceded by a distributed hash table
(DHT) prefix.}  While the fingerprint methods are not a significant
evolution over previous attacks, they reinforce the success these
methods bring to a closed network environment.  What is perhaps most
perhaps most noteworthy about their results is the overall rate of
success is still only approximately 80\% at peak against a total of less
than 500 \texttt{.onion} sites surveyed.  It would seem that with such a
small pool of candidate sites, it might be possible to obtain nearly
perfect fingerprint identification if not with past methods, with
tailored site-specific classifiers.  This suggests that Tor does a
reasonably good job of masking identifying details of system behavior.

Rounding out our literature survey on fingerprint attacks in the hidden
services domain is work by Panchenko et al.,
2017\cite{panchenko_analysis_2017}.  Here the authors outline a method
to first identify Tor hidden service communications, then once
identified, unveil the specific hidden service using a classifier with
the highest rate of success compared to other known classifiers in
testing.  A key contribution of this work is the observation that many
websites are loaded from multiple Tor circuits, some of which may use
entirely different Tor network entry nodes and consequently entirely
unique end-to-end Tor circuit paths.  Panchenko et al. are able to
achieve approximately twice the rate of fingerprint success compared to
previous fingerprinting efforts, a significant improvement.

\subsection{Network Services Fingerprinting}

While some studies such as in Feamster \& Dingledine 2004
\cite{feamster_location_2004} and Murdoch et al., 2007
\cite{murdoch_sampled_2007} have enumerated network-level threats
against the Tor relay network, few studies have examined the ability to
or the threat upon which Tor systems can be passively or actively
fingerprinted without direct observational access of the overlay
traffic.  In this section we highlight two such network-based
fingerprint attacks that are not on the data path between of a Tor
overlay circuit for any particular communicating target.  We will refer
to these ideas in the next section with our own contributions and
measurement work.

In Greschbach et al., 2016 \cite{greschbach_effect_2016} DNS queries
associated with Tor overlay traffic is considered in correlation
attacks.  That is, off-path networks and resolvers either alone or in
conjunction with the capability to observe Tor traffic can enhance
anonymity attacks.  Tor applications will resolve domain names using
local resolver policy at whatever exit node an application happens to be
using.  The DNS resolver policy is operator independent, but is often
whatever ISP or network default is in use.  Greschbach et al. find that
40\% of all exit nodes use Google public DNS.  Regardless of the
specific DNS resolver in use, these resolvers are by definition a key
component of each Tor user's communication path.  If the exit relay
traffic can be observed or even just the exit resolvers, DNS traffic
analysis may help uncover one or both ends of a Tor conversation.  The
authors use the DNS signaling to enhance traditional website
fingerprinting attacks, bringing additional insight to refine end point
identification.  Unfortunately the authors did not perform correlation
attacks on specific Tor usage traffic so we cannot directly compare
their results to other fingerprinting attacks.  They did find that they
are able to classify websites based on DNS requests with over 90\%
accuracy, but it is unclear if this enhances identification of specific
sessions or not.

\section{Our Contributions}\label{sec:our_contributions}

Many of the system fingerprint attacks are already well known threats,
but published evaluation of these attacks is relatively sparse.  In this
section we offer some evaluation of the Tor system through various means
of observational insight including off and on-path from the Tor overlay
traffic.  In the accompanying code repository we provide some sample
code, tools, and data to help illustrate this
work.\footnote{\url{https://github.com/jtkristoff/cs587}}

\subsection{X.509 Certificate Identification}

Practically all running Tor relay nodes use a version of code that
includes this from the \texttt{tor/tortls.c} source file:

\begin{footnotesize}
\begin{verbatim}
    tor_tls_init();
    nickname = crypto_random_hostname(8, 20, "www.", ".net");
  #ifdef DISABLE_V3_LINKPROTO_SERVERSIDE
    nn2 = crypto_random_hostname(8, 20, "www.", ".net");
  #else
    nn2 = crypto_random_hostname(8, 20, "www.", ".com");
  #endif
\end{verbatim}
\end{footnotesize}

This code snippet is responsible for generating the X.509 certificate
used by a Tor relay node for all the TLS-encrypted and encapsulated
traffic.  The effect of these instructions is that all Tor relay node
X.509 certificates will have the \emph{CN} field value set of the form
\texttt{www.[a-z2-7]\{8,20\}.net} and the \emph{ISSUER\_CN} field a
value of the form \texttt{www.[a-z2-7]8,20\}.com}.  The middle label in
these generated domain names are Base32 encoded and always between eight
and 20 characters in length.  This pattern of X.509 certificate
attributes is unlikely to be used by any system not based off the Tor
relay code.  Obtaining the X.509 certificates from every recently
reachable Tor node exhibited this certificate attribute pattern.  Where
this becomes slightly more useful is in detecting Tor nodes not listed
in the public directory.  The author had recently conducted
Internet-wide TCP destination port 443 scans and where X.509
certificates were found, hundreds of X.509 certificates matching these
attribute properties that were not listed in the public directory were
found over the course of a few months.  All of those discovered tested
positively as unpublished bridge nodes.

\subsection{Network Neighborhood}

The Tor relay directory includes BGP route origination information
including a source autonomous network name and number.  Evaluating the
ASNs in use by Tor relays can highlight the propensity for certain types
of networks or locations that run relays.  It may also suggest those
that are shy away from supporting Tor node relays, although this is
likely a large field since in total there are less than 7000 publicly
listed Tor nodes as of this writing.

Let us consider the following six classes of networks: 1. Broadband, 2.
Enterprise, 3. Datacenter/Cloud, 4. Content Delivery Network, 5. Tier 1/2 ISP,
and 6. Internet Exchange Point.  We can then classify the top locations where Tor relay
nodes are located to see if it any particular network type is more
popular for hosting Tor relay nodes than another:

\begin{center}
  \begin{table}[h]
  \begin{tabular} { l | c | r }
  \hline
  count & type & network \\ \hline
  563 & Datacenter/Cloud & OVH \\ \hline
  428 & Datacenter/Cloud & ONLINE S.A.S.  \\ \hline
  339 & Datacenter/Cloud & Hetzner \\ \hline
  215 & Datacenter/Cloud & Linode \\ \hline
  182 & Tier 1/2 ISP & Deutsche Telekom \\ \hline
  109 & UPC & Broadband \\ \hline
  106 & Datacenter/Cloud & DigitalOcean \\ \hline
  105 & Datacenter/Cloud & Choopa \\ \hline
  100 & Broadband & Broadband \\ \hline
  \hline
  \end{tabular}
  \caption{Tor relay host networks}
  \end{table}
\end{center}

The Datacenter/Cloud network environment appears to be a popular choice.
In fact the next two entries would have been sister networks to
DigitalOcean and most of the remaining top 20 most popular networks
would continue to be dominated by hosting providers.  We could also
evaluate the network location of relays along other lines such as IP
route prefix or geolocation.

\section{Future Work}\label{sec:future_work}

In the literature we evaluated there was no survey of originating Tor
exit node IP address usage.  It would be helpful for instance to
enumerate through a Tor directory, setting up a circuit with each
available exit node, contact a destination we can observe, and evaluate
the exit relay source IP address used.  With this data we could then
correlate if the exit node address selected as part of the circuit
construction is the same as seen by our control destination.  We
would expect this to be the case most of the time, but if not it may
present some interesting provisioning and addressing arrangements that
are not reflected in the published directories.  Conducting these
surveys over the course of time, perhaps with some form of traceroute
might also confirm the any one exit node has a stable path to our
observed destination or not.  We would also be interested to see the
propensity of Tor exit nodes that support IPv6 by default when we
attempt to access an IPv6 only destination.

When we consider the network neighborhood of Tor relay, bridge, and
hidden service nodes, we suspect some interesting patterns may arise for
some partial sets of nodes.  For instance, if we were worried about
malicious node relays, do certain networks seem more prone to hosting
them than other networks?  What about networks where we might consider
the trustworthiness of the operator open to question?  For instance, a
a government official in country X may be suspicious of any relay
operated by any organization affiliated with another country's
government or military.  Uncovering interesting hosting anomalies or
groupings may highlight nodes to prefer or avoid.

In the public Tor node directory listed each relay includes basic
operating system information, most of which identify as Linux.  We have
begun cataloging these systems with a more thorough service scan using
\emph{nmap} to better evaluate each relay's system profile to see if we
can draw any conclusions about listening services and operating system
configuration for typical Tor relay nodes.  We are particularly
interested in other application services that may be co-resident on
relay nodes.

Most known Tor relay nodes have or have had one or more assigned
associated public domain names.  Issuing historical queries for name
mappings on a recent Tor node directory list we try to observe trends or
patterns in naming schemes.\footnote{Passive DNS data obtained through a
research grant by Farsight Security
\url{https://www.farsightsecurity.com/}}.  We have not concluded our
analysis using passive DNS data, but we suspect this may be an area that
will lead to interesting correlations.  Upon a cursory look we find some
relay nodes thousands of name mappings to an IP address while others
have none.  This outliers may be artifacts of the particular relay
provider.  Other times where we find one or a small number of associated
names there may be an obvious association with the Tor system and other
times not.  it is not yet clear what patterns may emerge.

\section{Conclusion}

Website fingerprinting attacks have been a popular area of interest
throughout much of the research literature we have examined.
In laboratory settings these attacks often perform well, but rarely are
these attacks evaluated at scale.  Some Tor nodes see an enormous amount
of traffic and the variability of usage throughout the overlay network
is high.  This poses practical deployment challenges that has not been
well studied.  It is unclear how well these attacks ultimately fare in
the wild outside of controlled conditions.

Our initial exploration of system fingerprinting beyond entry and exit
traffic analysis suggests that the overlay network is both imminently
knowable and discoverable, perhaps more easily than some would have
imagined.  Being able to fingerprint and identify the unpublished
systems and users of Tor has ramifications for availability since
censors are often looking for easy ways to thwart its use.  Overall it
seems that there way significantly more avenues of fingerprint attacks
than defenders can likely keep up with.  The availability arms race for
anonymity systems seems likely to continue for years to come.

\bibliography{torfp}

\end{document}
